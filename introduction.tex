Ostatnie lata przynoszą coraz szybszy rozwój nurtu \emph{representation learning} \cite{Bengio2012}. Leżąca u jego podstaw idea twierdzi, że kluczem do skutecznej klasyfikacji trudnych zbiorów danych, jest odpowiednia ekstrakcja cech. Transformacja przestrzeni danych dostępnych w zbiorze do innej przestrzeni, w której samo zadanie trywializuje się i wymaga znacznie mniejszych zasobów i znacznie mniej złożonych metod do skutecznej klasyfikacji. 

Poniższy raport prezentuje rozszerzoną wersję publikowanych wcześniej badań dotyczących metody transformacji przestrzeni cech, inspirowanej intuicją fotograficzną, sugerującą, że możemy wykorzystać parę cech zbioru do stworzenia \emph{wirtualnej kliszy}, którą zamiast na światło, eksponujemy na wiązkę próbek. Stąd sama jej główna struktura nazywana jest \emph{ekspozerem}.

Tak stworzone struktury, można wykorzystywać zarówno jako klasyfikatory bazowe, jak i -- kiedy decydujemy się na zbiór podzbiorów cech -- komitety klasyfikatorów. Takie podejście jest, jak w większości metod obrazowych, łatwe do przystosowania do przetwarzania równoległego, jak i niezależne od wielkości próbki, odporne na zwiększenie liczby cech zbioru uczącego, jak i pozwala na regularyzację w przestrzeni cech, przeprowadzaną niezależnie od przestrzeni oryginalnego zbioru.

Proponowana metoda została przetestowana eksperymentalnie na wyborze dziesięciu zbiorów benchmarkowych, możliwie przedstawiającym przekrój problemów, dla których algorytm był projektowany. Testy zawierające się w poniższym raporcie mają na celu weryfikację, jak algorytm radzi sobie w podstawowej metodzie głosowania wewnątrz komitetu klasyfikatorów, w której każdy członek komitetu oddaje głos o tej samej wadze.