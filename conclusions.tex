
Powyższy raport prezentuje klasyfikator sformułowany jako komitet wielospektralnych obrazów generowanych na podstawie kombinowanych podzbiorów cech zbioru danych. Każdy obraz opisuje wybraną podprzestrzeń cech, a każda jego warstwa -- odnosi się do jednej klasy problemu. Podejście tego typu przejawia odporność na klątwę wymiarowości, co pozwala na efektywne przetwarzanie danych wielowymiarowych.

Głównym celem przeprowadzonych eksperymentów była ocena skuteczności algorytmu przy ograniczeniu jego parametrów jedynie do tych podstawowych. Ich wyniki wykazują znaczące pogorszenie jego jakości.

Sugeruje to, że w kolejnych badaniach należy podjąć próbę opracowania metody automatycznej selekcji parametrów, dopasowującej się do badanego problemu. 