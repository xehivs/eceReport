\begin{table*}[htb]
    \sisetup{round-mode=places}
    \sisetup{
    	round-precision = 3,
    	round-integer-to-decimal
    }
    \parbox{.65\linewidth}{
    
	    Tabele 14--25, wraz z wykresami, przedstawiają wyniki osiągnięte przez algorytm dla \emph{promienia} \oldstylenums9\% oraz \emph{limitu} \oldstylenums{15} i parametrów \emph{ziarna} z przedziału \oldstylenums{1}--\oldstylenums{19}\. Testy zostały przeprowadzone z wykorzystaniem walidacji krzyżowej \oldstylenums5v\oldstylenums2.
    	
    	W wypadku zbiorów zawierających problem wieloetykietowy, wartość \textsc{bac} została wyliczona jako uśrednienie z testów tego typu przeprowadzonych osobno dla każdej klasy względem konkatenacji próbek pozostałych klas.
    }\hfill
    \parbox{.32\linewidth}{
	    \begin{rideg}{products/g_australian.csv}{Australian}\end{rideg}
	}
\end{table*}

\begin{table*}[!ht]
    \sisetup{round-mode=places}
    \sisetup{
    	round-precision = 3,
    	round-integer-to-decimal
    }
    \parbox{.32\linewidth}{
	    \begin{rideg}{products/g_balance.csv}{Balance}\end{rideg}
	}
	\hfill
    \parbox{.32\linewidth}{
	    \begin{rideg}{products/g_breastcan.csv}{Breastcan}\end{rideg}
	}
	\hfill
    \parbox{.32\linewidth}{
	    \begin{rideg}{products/g_diabetes.csv}{Diabetes}\end{rideg}
	}
\end{table*}


\begin{table*}[!ht]
    \sisetup{round-mode=places}
    \sisetup{
    	round-precision = 3,
    	round-integer-to-decimal
    }
    \parbox{.32\linewidth}{
	    \begin{rideg}{products/g_german.csv}{German}\end{rideg}
	}
	\hfill
    \parbox{.32\linewidth}{
	    \begin{rideg}{products/g_hayes.csv}{Hayes}\end{rideg}
	}
	\hfill
    \parbox{.32\linewidth}{
	    \begin{rideg}{products/g_heart.csv}{Heart}\end{rideg}
	}
\end{table*}

\begin{table*}[!ht]
    \sisetup{round-mode=places}
    \sisetup{
    	round-precision = 3,
    	round-integer-to-decimal
    }
    \parbox{.32\linewidth}{
	    \begin{rideg}{products/g_ionosphere.csv}{Ionosphere}\end{rideg}
	}
	\hfill
    \parbox{.32\linewidth}{
	    \begin{rideg}{products/g_iris.csv}{Iris}\end{rideg}
	}
	\hfill
    \parbox{.32\linewidth}{
	    \begin{rideg}{products/g_soybean.csv}{Soybean}\end{rideg}
	}
\end{table*}

\FloatBarrier


\begin{table*}[!ht]
	\sisetup{round-mode=places}
    \sisetup{
		round-precision = 3,
		round-integer-to-decimal
	}
	
	 \parbox{.49\linewidth}{   
		\caption{Najlepsze \emph{ziarno} względem accuracy}
	    \label{tab:accuracy}
		\centering
		\begin{tabular}{lSS}
			\toprule
			{zbiór danych} & {ziarno} & {accuracy} \\\midrule
		    \csvreader[head to column names]{products/acc.csv}{}%
		    {\emph{\filename} & \graing & \accuracyg\\}
		\end{tabular}
	}
	\hfill
	\parbox{.49\linewidth}{
		\caption{Najlepsze \emph{ziarno} względem \textsc{bac}}
	    \label{tab:bac}
		\centering
		\begin{tabular}{lS>{\color{red}}S}
			\toprule
			{zbiór danych} & {ziarno} & {\textsc{bac}} \\\midrule
		    \csvreader[head to column names]{products/bac.csv}{}%
		    {\emph{\filename} & \graing & \bacg\\}
		\end{tabular}
	}
	
    \parbox{\linewidth}{
    	\begin{tikzpicture}
		\begin{axis}[
			    ybar,
			    width = \linewidth,
			    bar width=0.3cm,
			    height = 10cm,
			    symbolic x coords={australian, balance, breastcan, diabetes, german, hayes, heart, ionosphere, iris, soybean},			    
			    xtick=data,
			    grid=both,
			    grid style={line width=.1pt, draw=gray!10},
			    major grid style={line width=.2pt,draw=gray!50},
			    ymin=0, ymax=1,
				minor tick num=4,
			    ytick = {.5, .6, .7, .8, .9, 1},
			    yticklabels = { 50\%, 60\%, 70\%, 80\%, 90\%, 100\%},
				ticklabel style={font=\tiny,fill=white},				
			    x tick label style={rotate=30,anchor=east},	
		    ]
			\addplot[black, fill=white, postaction={pattern=north east lines}] table[x=filename, y=accuracyg, col sep=comma] {products/acc.csv};
			\addplot[red, fill=white, pattern color = red, postaction={pattern=north east lines}] table[x=filename, y=bacg, col sep=comma] {products/bac.csv};			
		\end{axis}
		\end{tikzpicture}
    }
\end{table*}

\FloatBarrier
