\documentclass[]{article}

\usepackage{polski,graphicx,comment,listings,url,colortbl,tikz}
\usepackage[utf8]{inputenc}
\usepackage{algorithm}
\usepackage{algpseudocode}
\usepackage{amsmath}
\usepackage{amssymb}
\usepackage{siunitx}
\usepackage{booktabs}
\usepackage{csvsimple}
\usepackage{pgfplots}
\usetikzlibrary{patterns}
\usepackage{placeins}
\usepackage{listings}
\newsavebox\lstbox

% Page & text layout
\usepackage{geometry}
\geometry{%
  a4paper,%
  top=2.5cm,%
  bottom=2.5cm,%
  left=2.5cm,%
  right=2.5cm%
}


\title{Raport 4 z ECE}
\author{Paweł Ksieniewicz}

\begin{document}

\newenvironment{ride}[2]
{
	\caption{Zbiór \emph{#2}}
    \label{tab:results}
	\centering
	\begin{tabular}{lS>{\color{red}}S}
		\\\toprule{promień} & {accuracy} & {\textsc{bac}} \\\midrule
	    \csvreader[head to column names]{#1}{}%
	    {\radius & \accuracy & \bac\\}%
	\end{tabular}
	
	\begin{tikzpicture}
	\begin{axis}[
	    grid=both,
	    grid style={line width=.1pt, draw=gray!10},
	    major grid style={line width=.2pt,draw=gray!50},
	    width=6cm,
	    height=8cm,
	    xmin=0.01, xmax=0.29,
	    ymin=.0, ymax=1,
	    ytick = {.5, .6, .7, .8, .9, 1},
	    yticklabels = { ~, 60\%, ~, 80\%, ~, 100\%},
	    xtick = {0, .05, .1, 0.15, 0.2, 0.25},
		xticklabels = {~, 5,10,15,20,25},
		xlabel = Promień,
		minor tick num=2,
		xlabel style={font=\tiny,fill=white},
		ticklabel style={font=\tiny,fill=white}		]
		\addplot[color=black] table [x=radius, y=accuracy, col sep=comma] {#1};
		\addplot[color=red] table [x=radius, y=bac, col sep=comma] {#1};
	\end{axis}
	\end{tikzpicture}
}{}

\newenvironment{rideg}[2]
{
	\caption{Zbiór \emph{#2}}
    \label{tab:results}
	\centering
	\begin{tabular}{lS>{\color{red}}S}
		\\\toprule{ziarno} & {accuracy} & {\textsc{bac}} \\\midrule
	    \csvreader[head to column names]{#1}{}%
	    {\grain & \accuracy & \bac\\}%
	\end{tabular}
	
	\begin{tikzpicture}
	\begin{axis}[
	    grid=both,
	    grid style={line width=.1pt, draw=gray!10},
	    major grid style={line width=.2pt,draw=gray!50},
	    width=6cm,
	    height=8cm,
	    xmin=1, xmax=19,
	    ymin=.0, ymax=1,
	    ytick = {.5, .6, .7, .8, .9, 1},
	    yticklabels = { ~, 60\%, ~, 80\%, ~, 100\%},
	    xtick = {1, 4, 7, 10, 13, 16, 19},
		xticklabels = {~, 4,7,10,13,16,~},
		xlabel = Promień,
		minor tick num=2,
		xlabel style={font=\tiny,fill=white},
		ticklabel style={font=\tiny,fill=white}		]
		\addplot[color=black] table [x=grain, y=accuracy, col sep=comma] {#1};
		\addplot[color=red] table [x=grain, y=bac, col sep=comma] {#1};
	\end{axis}
	\end{tikzpicture}
}{}


\newenvironment{ridel}[2]
{
	\caption{Zbiór \emph{#2}}
    \label{tab:results}
	\centering
	\begin{tabular}{lS>{\color{red}}S}
		\\\toprule{limit} & {accuracy} & {\textsc{bac}} \\\midrule
	    \csvreader[head to column names]{#1}{}%
	    {\limit & \accuracy & \bac\\}%
	\end{tabular}
	
	\begin{tikzpicture}
	\begin{axis}[
	    grid=both,
	    grid style={line width=.1pt, draw=gray!10},
	    major grid style={line width=.2pt,draw=gray!50},
	    width=6cm,
	    height=8cm,
	    xmin=1, xmax=19,
	    ymin=.0, ymax=1,
	    ytick = {.5, .6, .7, .8, .9, 1},
	    yticklabels = { ~, 60\%, ~, 80\%, ~, 100\%},
	    xtick = {1, 4, 7, 10, 13, 16, 19},
		xticklabels = {~, 4,7,10,13,16,~},
		xlabel = Limit,
		minor tick num=2,
		xlabel style={font=\tiny,fill=white},
		ticklabel style={font=\tiny,fill=white}		]
		\addplot[color=black] table [x=limit, y=accuracy, col sep=comma] {#1};
		\addplot[color=red] table [x=limit, y=bac, col sep=comma] {#1};
	\end{axis}
	\end{tikzpicture}
}{}

\begin{titlepage}

\begin{center}
	\large Politechnika Wrocławska\\
	Wydział Elektroniki\\
	Katedra Systemów i Sieci Komputerowych\\
	%W04/S-052/15
	NUMER RAPORTU	
\end{center}

\vspace*{7cm}
\hspace*{6cm}\parbox[p]{10cm}
{\large Badanie wpływu ograniczenia parametryzacji algorytmu Exposer Classifier Ensemble przy klasyfikacji danych wielowymiarowych}
\vspace*{1cm}

\hspace*{6cm}\large Paweł Ksieniewicz

\vspace*{4cm}

\hspace*{6cm}\parbox{10cm}{\large Słowa kluczowe: representation learning, zespoły klasyfikatorów, oprogramowanie, Python}

\vspace*{5cm}

\begin{center}
	Wrocław 2016
\end{center}
\end{titlepage}


\newpage

\section{Wstęp}
Ostatnie lata przynoszą coraz szybszy rozwój nurtu \emph{representation learning} \cite{Bengio2012}. Leżąca u jego podstaw idea twierdzi, że kluczem do skutecznej klasyfikacji trudnych zbiorów danych, jest odpowiednia ekstrakcja cech. Transformacja przestrzeni danych dostępnych w zbiorze do innej przestrzeni, w której samo zadanie trywializuje się i wymaga znacznie mniejszych zasobów i znacznie mniej złożonych metod do skutecznej klasyfikacji. 

Poniższy raport prezentuje rozszerzoną wersję publikowanych wcześniej badań dotyczących metody transformacji przestrzeni cech, inspirowanej intuicją fotograficzną, sugerującą, że możemy wykorzystać parę cech zbioru do stworzenia \emph{wirtualnej kliszy}, którą zamiast na światło, eksponujemy na wiązkę próbek. Stąd sama jej główna struktura nazywana jest \emph{ekspozerem}.

Tak stworzone struktury, można wykorzystywać zarówno jako klasyfikatory bazowe, jak i -- kiedy decydujemy się na zbiór podzbiorów cech -- komitety klasyfikatorów. Takie podejście jest, jak w większości metod obrazowych, łatwe do przystosowania do przetwarzania równoległego, jak i niezależne od wielkości próbki, odporne na zwiększenie liczby cech zbioru uczącego, jak i pozwala na regularyzację w przestrzeni cech, przeprowadzaną niezależnie od przestrzeni oryginalnego zbioru.

Proponowana metoda została przetestowana eksperymentalnie na wyborze dziesięciu zbiorów benchmarkowych, możliwie przedstawiającym przekrój problemów, dla których algorytm był projektowany. Testy zawierające się w poniższym raporcie mają na celu weryfikację, jak algorytm radzi sobie w podstawowej metodzie głosowania wewnątrz komitetu klasyfikatorów, w której każdy członek komitetu oddaje głos o tej samej wadze.

\section{Środowisko eksperymentów}
	\subsection{Zbiory danych}
	Aby zweryfikować skuteczność algorytmu, użyte zostało dziesięć baz benchmarkowych. Większość z nich pochodzi z bazy \textsc{ucimlr}\cite{asuncion2007uci} i zostały one wybrane tak, aby stworzyć możliwie różnorodną pulę problemów do rozwiązania. Selekcja została dokonana na bazie następujących wytycznych:

\begin{itemize}
	\item Zbiory powinny zawierać tak binarne (\emph{6 baz}) jak i wieloklasowe (\emph{4 bazy, po 3 i 4 klasy}) problemy do rozwiązania.
	\item Zbiory powinny być zarówno nisko, jak i wysokowymiarowe (\emph{od 4 do 35 cech}).
	\item Powinny być uwzględniane zarówno zbiory rzadkie (\emph{43 próbki dla bazy Soybean}) jak gęste (\emph{1000 próbek dla German}).
	\item Cechy analizowanych zbiorów powinny być tak ciągłe (\emph{Iris}) jak i dyskretne (\emph{Balance}).
\end{itemize}

Krótki przegląd wszystkich wybranych zbiorów zawiera się w Tabeli \ref{tab:datasets_overview}.

\begin{table*}[!ht]
	\caption{Przegląd zbiorów danych}
	\label{tab:datasets_overview}
	\centering
	\csvreader[tabular=lccc,
        table head=zbiór danych & l. cech & l. próbek & l. klas \\\midrule,
        late after line = \\]%
    {comparison.csv }{dataset=\dataset,attributes=\attributes,samples=\samples,classes=\classes}%
    {\emph{\dataset} & \attributes & \samples & \classes}%
\end{table*}
	
	\subsection{Narzędzia eksperymentalne}
	Eksperyment badawczy powstał w oparciu o autorski \emph{framework} uczenia maszynowego \textsc{ksskml}\footnote{\url{https://github.com/w4k2/KSSKML}} i został zaimplementowany jako jego rozszerzenie, udostępnione w repozytorium paczek \verb|pip| jako moduł \textsc{ece}\footnote{\url{https://github.com/w4k2/ece}}, aktualnie w wersji \oldstylenums0.\oldstylenums6.\oldstylenums3.

Zarówno framework \textsc{ksskml} jak i sam moduł \textsc{ece} są przedmiotem stałego rozwoju, a aby zapewnić ich poprawność i jakość kodu, każda aktualizacja jest weryfikowana przez zbiór testów jednostkowych i narzędzie badania jakości kodu. Współczynnik jakości kodu \textsc{gpa}, określany w skali od 0 do 4, dla frameworku wynosi 3.01, a dla modułu 3.33.

Listingi 1 oraz 2 pokazują stopień pokrycia przez testy kodu utrzymywanego dla prowadzenia eksperymentów na algorytmie \textsc{ece}.

\begin{lstlisting}[frame=single,caption=Pokrycie testów dla modułu \textsc{ece}]
XML: ~/dev/ece/nosetests.xml
Name             Stmts   Miss  Cover
------------------------------------
ece/ECE.py          61      0   100%
ece/Exposer.py     175      0   100%
ece.py               2      0   100%
------------------------------------
TOTAL              238      0   100%

Ran 4 tests in 22.902s
\end{lstlisting}

\begin{lstlisting}[frame=single,caption=Pokrycie testów dla frameworku \textsc{ksskml}]
XML: ~/dev/ksskml/nosetests.xml
Name                   Stmts   Miss  Cover
------------------------------------------
ksskml/Classifier.py      10      2    80%
ksskml/Dataset.py        103      1    99%
ksskml/Ensemble.py        12      3    75%
ksskml/KNN.py             30      0   100%
ksskml/Sample.py          16      0   100%
ksskml.py                  5      0   100%
ksskml/utils.py            9      1    89%
------------------------------------------
TOTAL                    185      7    96%

Ran 3 tests in 66.848s
\end{lstlisting}

Kod odpowiedzialny za przeprowadzenie eksperymentów zawartych w niniejszym raporcie, podsumowanie ich oraz sam tekst raportu zostały umieszczone w repozytorium w serwisie Github\footnote{\url{https://github.com/xehivs/eceReport}}. Listing 3 przedstawia kod przykładowego eksperymentu.

\begin{lstlisting}[frame=single,language=Python,caption=Kod przykładowego eksperymentu]
#!/usr/bin/env python
# -*- coding: utf-8 -*-
from ece import *
from ksskml import *

import sys
import csv

# Load dataset
datafile = sys.argv[1]
dbname = datafile.split('/')
dbname = dbname[len(dbname) - 1]
resampling = 200
if len(sys.argv) > 2:
    resampling = int(sys.argv[2])
dataset = Dataset(datafile)

# Predefined parameters
parameters = {
    'heart': {
        'radius': 21,
        'grain': 5
    },...
    'diabetes': {
        'radius': 23,
        'grain': 13
    }
}

# Parameters to test
dbn = dbname[:-4]

radiuses = [parameters[dbn]['radius']]
grains = [parameters[dbn]['grain']]
folds = xrange(0, 5)
approaches = [ECEApproach.random]
limits = xrange(1, 20, 2)
votingMethods = [ExposerVotingMethod.lone]
i = 0
amount = len(folds) * len(approaches) * \
    len(votingMethods) * len(grains) * len(radiuses) * len(limits)
print "\t%i instances to process" % amount

with open('results/l_%s' % dbname, 'wb') as csvfile:
    writer = csv.writer(csvfile, delimiter=',')
    headers = ['fold', 'radius', 'grain', 'limit', 'accuracy', 'bac']
    writer.writerow(headers)
    for fold in folds:
        dataset.setCV(fold)
        for approach in approaches:
            for limit in limits:
                for votingMethod in votingMethods:
                    for grain in grains:
                        for radius in radiuses:
                            dataset.clearSupports()

                            fRadius = radius / 100.

                            configuration = {
                                'radius': radius,
                                'grain': grain,
                                'limit': limit,
                                'dimensions': [2],
                                'eceApproach': approach,
                                'exposerVotingMethod': votingMethod
                            }

                            ensemble = ECE(dataset, configuration)
                            ensemble.learn()
                            ensemble.predict()
                            scores = dataset.score()

                            entry = {
                                "dataset": dbname,
                                "fold": fold,
                            }

                            entry.update(configuration)
                            entry.update(scores)

                            row = [fold, fRadius, grain, limit,
                                entry['accuracy'], entry['bac']]
                            print row
                            writer.writerow(row)
                            i += 1
\end{lstlisting}

\section{Eksperymenty}

Algorytm \textsc{ece} charakteryzuje się wysoką parametryzacją. Jednostkowy ekspozer opisywany jest przez jego głębię przestrzenną (liczbę cech użytych do jego konstrukcji), promień oddziaływania próbek oraz ziarno kwantyzacji. Dobór ekspozerów do budowy komitetu może być prowadzony na jeden z trzech sposobów:
	\begin{itemize}
		\item kosztowny czasowo przegląd zupełny, 
		\item nie dająca gwarancji poprawnej selekcji, ale szybka metoda losowych podzbiorów z limitem, oraz
		\item kompromis w postaci heurystycznej metody z parametrem puli analizowanych ekspozerów oraz limitu członków końcowego komitetu.
	\end{itemize}
	
Dodatkowo zaimplementowano w nim pięć rodzajów wag, wpływających na udział pojedynczego ekspozera w klasyfikatorze. Zakładając, że każdy składowy ekspozer jest określony przez takie same parametry wewnętrzne (co jest jedynie najprostszą jego odmianą), w obrębie eksperymentu mamy aż pięć zmiennych do przebadania, co w wypadku danych wielowymiarowych jest bardzo czasochłonne i zużywa bardzo duże ilości zasobów.

W związku z tym, poniższe eksperymenty mają za zadanie określić, jak algorytm radzi sobie w warunkach ograniczonej parametryzacji. We wszystkich z nich została więc przyjęta metoda doboru klasyfikatorów do komitetu, oparta na losowych podzbiorach z limitem, a głosowanie odbywa się przy użyciu najprostszej metody, gdzie każdy klasyfikator bazowy ma taki sam wpływ na rezultat klasyfikacji. W oparciu o to założenie, zaproponowane zostały trzy eksperymenty:

\begin{itemize}
	\item Badanie wpływu na jakość klasyfikacji wartości promienia --- przy stałym ziarnie i limicie o wartości \verb|15|,
	\item Badanie wpływu na jakość klasyfikacji wartości ziarna --- przy stałym promieniu o wartości \verb|9|\% i limicie o wartości \verb|15|,
	\item Badanie wpływu na jakość klasyfikacji wartości limitu --- przy wartościach promienia i ziarna, które dały najlepsze rezultaty w poprzednich eksperymentach. 
\end{itemize}

	\subsection{Badanie wpływu promienia}
	%\begin{table*}[!ht]
    \sisetup{round-mode=places}
    \sisetup{
    	round-precision = 3,
    	round-integer-to-decimal
    }
    \parbox{.65\linewidth}{
    	Tabele 2--13, wraz z wykresami, przedstawiają wyniki osiągnięte przez algorytm dla \emph{ziarna} oraz \emph{limitu} \oldstylenums{15} i parametrów \emph{promienia} z przedziału \oldstylenums{1}--\oldstylenums{29}\%. Testy zostały przeprowadzone z wykorzystaniem walidacji krzyżowej \oldstylenums5v\oldstylenums2.
    	
    	W wypadku zbiorów zawierających problem wieloetykietowy, wartość \textsc{bac} została wyliczona jako uśrednienie z testów tego typu przeprowadzonych osobno dla każdej klasy względem konkatenacji próbek pozostałych klas.
	}
	\hfill
    \parbox{.32\linewidth}{
	    \begin{ride}{products/r_soybean.csv}{Soybean}\end{ride}
	}
\end{table*}

\begin{table*}[htb]
    \sisetup{round-mode=places}
    \sisetup{
    	round-precision = 3,
    	round-integer-to-decimal
    }
    \parbox{.32\linewidth}{
	    \begin{ride}{products/r_australian.csv}{Australian}\end{ride}
	}
	\hfill
    \parbox{.32\linewidth}{
	    \begin{ride}{products/r_balance.csv}{Balance}\end{ride}
	}
	\hfill
    \parbox{.32\linewidth}{
	    \begin{ride}{products/r_breastcan.csv}{Breastcan}\end{ride}
	}

\end{table*}


\begin{table*}[!ht]
    \sisetup{round-mode=places}
    \sisetup{
    	round-precision = 3,
    	round-integer-to-decimal
    }
    \parbox{.32\linewidth}{
	    \begin{ride}{products/r_diabetes.csv}{Diabetes}\end{ride}
	}
	\hfill
    \parbox{.32\linewidth}{
	    \begin{ride}{products/r_german.csv}{German}\end{ride}
	}
	\hfill
    \parbox{.32\linewidth}{
	    \begin{ride}{products/r_hayes.csv}{Hayes}\end{ride}
	}
\end{table*}

\begin{table*}[!ht]
    \sisetup{round-mode=places}
    \sisetup{
    	round-precision = 3,
    	round-integer-to-decimal
    }
    \parbox{.32\linewidth}{
	    \begin{ride}{products/r_heart.csv}{Heart}\end{ride}
	}
	\hfill
    \parbox{.32\linewidth}{
	    \begin{ride}{products/r_ionosphere.csv}{Ionosphere}\end{ride}
	}
	\hfill
    \parbox{.32\linewidth}{
	    \begin{ride}{products/r_iris.csv}{Iris}\end{ride}
	}
\end{table*}

\begin{table*}[!ht]
	\sisetup{round-mode=places}
    \sisetup{
		round-precision = 3,
		round-integer-to-decimal
	}
    \parbox{.49\linewidth}{   
		\caption{Najlepszy \emph{promień} względem accuracy}
	    \label{tab:accuracy}
		\centering
		\begin{tabular}{lSS}
			\toprule
			{zbiór danych} & {promień} & {accuracy} \\\midrule
		    \csvreader[head to column names]{products/acc.csv}{}%
		    {\emph{\filename} & \radiusr & \accuracyr\\}
		\end{tabular}
	}
	\hfill
	\parbox{.49\linewidth}{
		\caption{Najlepszy \emph{promień} względem \textsc{bac}}
	    \label{tab:bac}
		\centering
		\begin{tabular}{lS>{\color{red}}S}
			\toprule
			{zbiór danych} & {promień} & {\textsc{bac}} \\\midrule
		    \csvreader[head to column names]{products/bac.csv}{}%
		    {\emph{\filename} & \radiusr & \bacr\\}
		\end{tabular}
	}
	
    \parbox{\linewidth}{
    	\begin{tikzpicture}
		\begin{axis}[
			    ybar,
			    width = \linewidth,
			    bar width=0.3cm,
			    height = 10cm,
			    symbolic x coords={australian, balance, breastcan, diabetes, german, hayes, heart, ionosphere, iris, soybean},			    
			    xtick=data,
			    grid=both,
			    grid style={line width=.1pt, draw=gray!10},
			    major grid style={line width=.2pt,draw=gray!50},
			    ymin=0, ymax=1,
				minor tick num=4,
			    ytick = {.5, .6, .7, .8, .9, 1},
			    yticklabels = { 50\%, 60\%, 70\%, 80\%, 90\%, 100\%},
				ticklabel style={font=\tiny,fill=white},				
			    x tick label style={rotate=30,anchor=east},	
		    ]
			\addplot[black, fill=white, postaction={pattern=north east lines}] table[x=filename, y=accuracyr, col sep=comma] {products/acc.csv};
			\addplot[red, fill=white, pattern color = red, postaction={pattern=north east lines}] table[x=filename, y=bacr, col sep=comma] {products/bac.csv};
		\end{axis}
		\end{tikzpicture}
    }
\end{table*}

\FloatBarrier
	
	\subsection{Badanie wpływu ziarna}
	%\begin{table*}[htb]
    \sisetup{round-mode=places}
    \sisetup{
    	round-precision = 3,
    	round-integer-to-decimal
    }
    \parbox{.65\linewidth}{
    
	    Tabele 14--25, wraz z wykresami, przedstawiają wyniki osiągnięte przez algorytm dla \emph{promienia} \oldstylenums9\% oraz \emph{limitu} \oldstylenums{15} i parametrów \emph{ziarna} z przedziału \oldstylenums{1}--\oldstylenums{19}\. Testy zostały przeprowadzone z wykorzystaniem walidacji krzyżowej \oldstylenums5v\oldstylenums2.
    	
    	W wypadku zbiorów zawierających problem wieloetykietowy, wartość \textsc{bac} została wyliczona jako uśrednienie z testów tego typu przeprowadzonych osobno dla każdej klasy względem konkatenacji próbek pozostałych klas.
    }\hfill
    \parbox{.32\linewidth}{
	    \begin{rideg}{products/g_australian.csv}{Australian}\end{rideg}
	}
\end{table*}

\begin{table*}[!ht]
    \sisetup{round-mode=places}
    \sisetup{
    	round-precision = 3,
    	round-integer-to-decimal
    }
    \parbox{.32\linewidth}{
	    \begin{rideg}{products/g_balance.csv}{Balance}\end{rideg}
	}
	\hfill
    \parbox{.32\linewidth}{
	    \begin{rideg}{products/g_breastcan.csv}{Breastcan}\end{rideg}
	}
	\hfill
    \parbox{.32\linewidth}{
	    \begin{rideg}{products/g_diabetes.csv}{Diabetes}\end{rideg}
	}
\end{table*}


\begin{table*}[!ht]
    \sisetup{round-mode=places}
    \sisetup{
    	round-precision = 3,
    	round-integer-to-decimal
    }
    \parbox{.32\linewidth}{
	    \begin{rideg}{products/g_german.csv}{German}\end{rideg}
	}
	\hfill
    \parbox{.32\linewidth}{
	    \begin{rideg}{products/g_hayes.csv}{Hayes}\end{rideg}
	}
	\hfill
    \parbox{.32\linewidth}{
	    \begin{rideg}{products/g_heart.csv}{Heart}\end{rideg}
	}
\end{table*}

\begin{table*}[!ht]
    \sisetup{round-mode=places}
    \sisetup{
    	round-precision = 3,
    	round-integer-to-decimal
    }
    \parbox{.32\linewidth}{
	    \begin{rideg}{products/g_ionosphere.csv}{Ionosphere}\end{rideg}
	}
	\hfill
    \parbox{.32\linewidth}{
	    \begin{rideg}{products/g_iris.csv}{Iris}\end{rideg}
	}
	\hfill
    \parbox{.32\linewidth}{
	    \begin{rideg}{products/g_soybean.csv}{Soybean}\end{rideg}
	}
\end{table*}

\FloatBarrier


\begin{table*}[!ht]
	\sisetup{round-mode=places}
    \sisetup{
		round-precision = 3,
		round-integer-to-decimal
	}
	
	 \parbox{.49\linewidth}{   
		\caption{Najlepsze \emph{ziarno} względem accuracy}
	    \label{tab:accuracy}
		\centering
		\begin{tabular}{lSS}
			\toprule
			{zbiór danych} & {ziarno} & {accuracy} \\\midrule
		    \csvreader[head to column names]{products/acc.csv}{}%
		    {\emph{\filename} & \graing & \accuracyg\\}
		\end{tabular}
	}
	\hfill
	\parbox{.49\linewidth}{
		\caption{Najlepsze \emph{ziarno} względem \textsc{bac}}
	    \label{tab:bac}
		\centering
		\begin{tabular}{lS>{\color{red}}S}
			\toprule
			{zbiór danych} & {ziarno} & {\textsc{bac}} \\\midrule
		    \csvreader[head to column names]{products/bac.csv}{}%
		    {\emph{\filename} & \graing & \bacg\\}
		\end{tabular}
	}
	
    \parbox{\linewidth}{
    	\begin{tikzpicture}
		\begin{axis}[
			    ybar,
			    width = \linewidth,
			    bar width=0.3cm,
			    height = 10cm,
			    symbolic x coords={australian, balance, breastcan, diabetes, german, hayes, heart, ionosphere, iris, soybean},			    
			    xtick=data,
			    grid=both,
			    grid style={line width=.1pt, draw=gray!10},
			    major grid style={line width=.2pt,draw=gray!50},
			    ymin=0, ymax=1,
				minor tick num=4,
			    ytick = {.5, .6, .7, .8, .9, 1},
			    yticklabels = { 50\%, 60\%, 70\%, 80\%, 90\%, 100\%},
				ticklabel style={font=\tiny,fill=white},				
			    x tick label style={rotate=30,anchor=east},	
		    ]
			\addplot[black, fill=white, postaction={pattern=north east lines}] table[x=filename, y=accuracyg, col sep=comma] {products/acc.csv};
			\addplot[red, fill=white, pattern color = red, postaction={pattern=north east lines}] table[x=filename, y=bacg, col sep=comma] {products/bac.csv};			
		\end{axis}
		\end{tikzpicture}
    }
\end{table*}

\FloatBarrier

	
	\subsection{Badanie wpływu limitu}
	%	
\begin{table*}[htb]
    \sisetup{round-mode=places}
    \sisetup{
    	round-precision = 3,
    	round-integer-to-decimal
    }
     \parbox{.65\linewidth}{
    
	    Tabele 26--37, wraz z wykresami, przedstawiają wyniki osiągnięte przez algorytm dla najlepszego promienia oraz ziarna z poprzednich eksperymentów i parametrów \emph{limitu} z przedziału \oldstylenums{1}--\oldstylenums{19}\. Testy zostały przeprowadzone z wykorzystaniem walidacji krzyżowej \oldstylenums5v\oldstylenums2.
    	
    	W wypadku zbiorów zawierających problem wieloetykietowy, wartość \textsc{bac} została wyliczona jako uśrednienie z testów tego typu przeprowadzonych osobno dla każdej klasy względem konkatenacji próbek pozostałych klas.
    }\hfill
    \parbox{.32\linewidth}{
	    \begin{ridel}{products/l_australian.csv}{Australian}\end{ridel}
	}
\end{table*}

\begin{table*}[!ht]
    \sisetup{round-mode=places}
    \sisetup{
    	round-precision = 3,
    	round-integer-to-decimal
    }
    
    \parbox{.32\linewidth}{
	    \begin{ridel}{products/l_balance.csv}{Balance}\end{ridel}
	}
	\hfill
    \parbox{.32\linewidth}{
	    \begin{ridel}{products/l_breastcan.csv}{Breastcan}\end{ridel}
	}
	\hfill
    \parbox{.32\linewidth}{
	    \begin{ridel}{products/l_diabetes.csv}{Diabetes}\end{ridel}
	}
\end{table*}


\begin{table*}[!ht]
    \sisetup{round-mode=places}
    \sisetup{
    	round-precision = 3,
    	round-integer-to-decimal
    }
    \parbox{.32\linewidth}{
	    \begin{ridel}{products/l_german.csv}{German}\end{ridel}
	}
	\hfill
    \parbox{.32\linewidth}{
	    \begin{ridel}{products/l_hayes.csv}{Hayes}\end{ridel}
	}
	\hfill
    \parbox{.32\linewidth}{
	    \begin{ridel}{products/l_heart.csv}{Heart}\end{ridel}
	}
\end{table*}

\begin{table*}[!ht]
    \sisetup{round-mode=places}
    \sisetup{
    	round-precision = 3,
    	round-integer-to-decimal
    }
    \parbox{.32\linewidth}{
	    \begin{ridel}{products/l_ionosphere.csv}{Ionosphere}\end{ridel}
	}
	\hfill
    \parbox{.32\linewidth}{
	    \begin{ridel}{products/l_iris.csv}{Iris}\end{ridel}
	}
	\hfill
    \parbox{.32\linewidth}{
	    \begin{ridel}{products/l_soybean.csv}{Soybean}\end{ridel}
	}
\end{table*}

\begin{table*}[!ht]
	\sisetup{round-mode=places}
    \sisetup{
		round-precision = 3,
		round-integer-to-decimal
	}
    \parbox{.45\linewidth}{   
		\caption{Najlepszy \emph{promień} względem accuracy}
	    \label{tab:accuracy}
		\centering
		\begin{tabular}{lSS}
			\toprule
			{zbiór danych} & {limit} & {accuracy} \\\midrule
		    \csvreader[head to column names]{products/acc.csv}{}%
		    {\emph{\filename} & \limitl & \accuracyl\\}
		\end{tabular}
	}
	\hfill
	\parbox{.45\linewidth}{
		\caption{Najlepszy \emph{promień} względem \textsc{bac}}
	    \label{tab:bac}
		\centering
		\begin{tabular}{lS>{\color{red}}S}
			\toprule
			{zbiór danych} & {limit} & {\textsc{bac}} \\\midrule
		    \csvreader[head to column names]{products/bac.csv}{}%
		    {\emph{\filename} & \limitl & \bacl\\}
		\end{tabular}
	}
	
    \parbox{\linewidth}{
    	\begin{tikzpicture}
		\begin{axis}[
			    ybar,
			    width = \linewidth,
			    bar width=0.3cm,
			    height = 10cm,
			    symbolic x coords={australian, balance, breastcan, diabetes, german, hayes, heart, ionosphere, iris, soybean},			    
			    xtick=data,
				minor tick num=4,
			    grid=both,
			    grid style={line width=.1pt, draw=gray!10},
			    major grid style={line width=.2pt,draw=gray!50},
			    ymin=0, ymax=1,
			    ytick = {.5, .6, .7, .8, .9, 1},
			    yticklabels = { 50\%, 60\%, 70\%, 80\%, 90\%, 100\%},
				ticklabel style={font=\tiny,fill=white},				
			    x tick label style={rotate=30,anchor=east},	
		    ]
			\addplot[black, fill=white, postaction={pattern=north east lines}] table[x=filename, y=accuracyl, col sep=comma] {products/acc.csv};
			\addplot[red, fill=white, pattern color = red, postaction={pattern=north east lines}] table[x=filename, y=bacl, col sep=comma] {products/bac.csv};
		\end{axis}
		\end{tikzpicture}
    }
\end{table*}

\FloatBarrier
	
	\subsection{Porównanie ze standardowymi klasyfikatorami bazowymi}
	\FloatBarrier
\begin{table}[ht]
	\caption{Podsumowanie rezultatów}
	\vspace{.5cm}
    \label{tab:podsumowanie}
	\centering
	\begin{tabular}{l>{\color{red}}ccccc}
		& \textsc{ece} & \textsc{svm} & k-\textsc{nn} & \textsc{melm} & \textsc{fa}\\
		\multicolumn{6}{l}{\tiny \bfseries problemy binarne}\\
		\emph{Australian} &0.863 & 0.820 & 0.849 & \textbf{0.866} & 0.847\\
		\emph{Breastcan} & \textbf{0.978} & 0.926 & 0.976 & 0.976 & 0.973\\
		\emph{Diabetes} & 0.733  & \textbf{0.796} & 0.792 & 0.744 & 0.682\\
		\emph{German} & 0.673 & \textbf{0.850} & 0.757 & 0.705 & 0.588\\
		\emph{Heart} & 0.816 & 0.770 & 0.806 & \textbf{0.831} & 0.792\\
		\emph{Ionosphere} & 0.763 & --- & --- & \textbf{0.892} & 0.794\\\\
		\multicolumn{6}{l}{\tiny \bfseries problemy wieloetykietowe}\\
		\emph{Balance} & 0.754 & --- & \textbf{0.876} & --- & ---\\		
		\emph{Hayes} & \textbf{0.663} & --- & 0.620 & --- & ---\\
		\emph{Iris} & 0.963 & --- & \textbf{0.967} & --- & ---\\
		\emph{Soybean} & 0.939 & --- & \textbf{1.000} & --- & ---\\
	\end{tabular}
\end{table}

Tabela \ref{tab:podsumowanie} prezentuje zestawienie najlepszych wyników z trzech przeprowadzonych eksperymentów i wyników dla konkurencyjnych klasyfikatorów bazowych, w podziale na problemy binarne oraz wieloetykietowe.

\section{Wnioski}
%
Powyższy raport prezentuje klasyfikator sformułowany jako komitet wielospektralnych obrazów generowanych na podstawie kombinowanych podzbiorów cech zbioru danych. Każdy obraz opisuje wybraną podprzestrzeń cech, a każda jego warstwa -- odnosi się do jednej klasy problemu. Podejście tego typu przejawia odporność na klątwę wymiarowości, co pozwala na efektywne przetwarzanie danych wielowymiarowych. Algorytm bez przeciwskazań może być używany do rozwiązywania realnych problemów.


\bibliographystyle{unsrt}
\bibliography{biblography}

\end{document}
